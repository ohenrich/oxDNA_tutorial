% General settings
\documentclass[slidestop,compress,9pt]{beamer}
\usetheme{AnnArbor}
\usepackage[UKenglish]{babel}
\usepackage[UKenglish]{isodate}
\usepackage[utf8]{inputenc}
\usepackage{hyperref}
\definecolor{links}{HTML}{2A1B81}
\hypersetup{colorlinks=true,allcolors=links}

% Code listings and pseudocode
\usepackage{listings}
\usepackage{algpseudocode}
\usepackage{algorithm}

% Mathematical formulas
\usepackage{amsmath}
\usepackage{mathtools}
\usepackage{bm}
\setcounter{MaxMatrixCols}{20}

% Graphics and figures
\usepackage{graphicx}
\graphicspath{{figures/}}
\setbeamertemplate{caption}[numbered]
\usepackage{fancybox}
\usepackage{pgfplots}

% Tikz
\usepackage{tikz}
\usepackage{tikz-3dplot}
\usetikzlibrary{shapes.geometric, arrows}

% For redefinition of texttt
%\let\textttorig\texttt
\usepackage[scaled=1.0]{couriers}

\setbeamertemplate{frametitle}{\thesection.~\insertsection\hspace*{0.45cm}\insertframetitle}

% Numbered sections
\setbeamertemplate{section in toc}[sections numbered]
\setbeamertemplate{subsection in toc}[subsections numbered]
\setbeamertemplate{subsubsection in toc}[subsubsections numbered]

% Listings
\definecolor{cident}{rgb}{0.0,0.0,0.0}
\definecolor{ckeyw}{rgb}{0,0,0.8}
\definecolor{ccomm}{rgb}{0,0.8,0}
\definecolor{cstr}{rgb}{0.8,0,0}
\definecolor{myyellow}{rgb}{0.99,0.76, 0.0}
\definecolor{mymagenta}{rgb}{1.0, 0.0, 1.0}

\lstset{language=C,
  basicstyle=\small\ttfamily,
  keywordstyle=\color{ckeyw}\bfseries,
  identifierstyle=\color{cident}\bfseries,
  commentstyle=\color{ccomm},
  stringstyle=\color{cstr},
  showstringspaces=false,
  breaklines=true,
  breakatwhitespace=true,
  tabsize=2,
  mathescape = false,
  columns=flexible,
  escapeinside={<@}{@>}
%   numbers=left,
%   stepnumber=1,
%   firstnumber=1,
%   numberfirstline=true,
  }

\title[The oxDNA Coarse-Grained Model of DNA and RNA]{\textbf{The oxDNA Coarse-Grained Model of DNA}}
\subtitle{An Introduction to the Model and Software Framework }
\author[Oliver Henrich]{Oliver Henrich \\ \small Email: \href{mailto:oliver.henrich@strath.ac.uk}{oliver.henrich@strath.ac.uk}}
\institute[U Strathclyde, Glasgow, UK]{Department of Physics, University of Strathclyde, Glasgow, UK}
\date{22nd September 2022}


\begin{document}

%\renewcommand<>{\texttt}[1]{%
%  \only#2{\textttorig{#1}}%
%}

\renewcommand<>{\texttt}{\only#1{\beameroriginal{\texttt}}}

% ==============================================================
% --- Welcome frame
% ==============================================================
\begin{frame}[plain]
\maketitle
\end{frame}

% ==============================================================
% --- TOC
% ==============================================================

\begin{frame}
\Large Outline
\normalsize
\tableofcontents
\end{frame}

% ==============================================================
% --- Content
% ==============================================================

\section{oxDNA Model}

\begin{frame}
\frametitle{DNA Facts}

\begin{itemize}
\item Human genome contains $3\times10^{9}$ base pairs (bps)
\item DNA loop around a nucleosome core particle contains 147 bps
\item Smallest loop in chromatin fibre consists of $5\times10^4$ bps
\item Atomistic simulation of DNA can model 3,000 bps (probably less) and typically resolve times on the $\mu$s-scale  
\item \textbf{Coarse-grained models} target much \textbf{larger time and length scales} in the range of ms and Mbps 
\end{itemize}

\begin{center}
\includegraphics[width=0.54\textwidth]{fromDNAtoChromatin.png}
\includegraphics[width=0.45\textwidth]{A-B-Z-DNA.png}\\
\textit{From DNA to chromatin} \hspace{2.75cm} \textit{A-DNA, B-DNA, Z-DNA}
\end{center}

\end{frame}

\begin{frame}
\frametitle{Overview}

\begin{itemize}
\item \textbf{Each nucleotide} is described as \textbf{rigid body}
\item \textbf{3 interaction} sites for backbone, stacking and hydrogen-bonding
\item \textbf{7 effective interactions} between nucleotides
\begin{itemize}
\item Bonded interaction for backbone connectivity
\item 6 pair interactions for excluded volume, stacking, cross-stacking, coaxial stacking, hydrogen-bonding and electrostatic interaction
\end{itemize}
\item \textbf{oxDNA: 13 DOF per nucleotide}\\
3 positions, 3 translational momenta, 3 angular momenta, 1 unit quaternion (4 components)
\item \textbf{Atomistic simulation} (pyrimidine base plus sugar-phosphate group):\\
around \textbf{200 DOF per nucleotide}\\
34 atoms per nucleotide, each with 3 positions and 3 momenta
\end{itemize}

\vspace*{-0.5cm}

\begin{columns}
\begin{column}{0.4\textwidth}
\begin{center}
\includegraphics[width=0.9\textwidth]{nucleotide_duplex_b.pdf}
\end{center}
\end{column}

\begin{column}{0.45\textwidth}
\begin{center}
\vspace*{0.5cm}
\begin{itemize}
\setlength\itemsep{7pt}
\item[] \textit{(a) oxDNA nucleotide}
\item[] \textit{(b) Duplex}
\item[] \textit{(c) Interactions}
\end{itemize}
\end{center}
\end{column}
\end{columns}

\end{frame}

\begin{frame}
\frametitle{Nucleotide Geometry}

\begin{columns}
\begin{column}{0.56\textwidth}
\begin{itemize}
\setlength\itemsep{20pt}
\item Each nucleotide has a centre of mass (COM) $\bm{r}_{COM}$, a base vector $\bm{b}$, base normal $\bm{n}$ and a third vector $\bm{y}=\bm{n}\times\bm{b}$
\item The \textbf{backbone interaction} site is at\\
$\bm{r}_{back}=\bm{r}_{COM} - 0.4\,\bm{b}$ \hspace{1.7cm}(oxDNA1)\\
$\bm{r}_{back}=\bm{r}_{COM} - 0.34\, \bm{b} + 0.3408\,\bm{y}$ (oxDNA2)
\item The \textbf{stacking interaction} site is at\\
$\bm{r}_{stack}=\bm{r}_{COM} + 0.34\, \bm{b}$
\item The \textbf{hydrogen-bonding interaction} site is at\\
$\bm{r}_{base}=\bm{r}_{COM} + 0.4\, \bm{b}$
\end{itemize}
\end{column}

\begin{column}{0.44\textwidth}
\vspace*{-0.5cm}
\begin{center}
\includegraphics[width=0.85\textwidth]{oxdna_oxdna2.png}\\
\end{center}
\textit{(a) oxDNA1 and oxDNA2 nucleotides: the base vector $\bm{b}$ is horizontally oriented from left to right, whereas the base normal $\bm{n}$ points away from the observer.}\\[3pt]
\textit{(b) The angled backbone interaction sites leads to the correct geometry with major and minor grooves.} 
\end{column}
\end{columns}

\end{frame}

\begin{frame}
\frametitle{Angles and Vectors}

\begin{columns}
\begin{column}{0.56\textwidth}
\textbf{Relative distance vectors} are defined between two nucleotides $i$ and $j$
\begin{itemize}
\setlength\itemsep{3pt}
\item backbone interaction sites\\
$\bm{r}_{back, ij}=\bm{r}_{back,i} - \bm{r}_{back,j}$
\item stacking interaction sites\\
$\bm{r}_{stack, ij}=\bm{r}_{stack,i} - \bm{r}_{stack,j}$
\item hydrogen-bonding interaction sites\\
$\bm{r}_{base, ij}=\bm{r}_{base,i} - \bm{r}_{base,j}$
\item mixed sites\\
$\bm{r}_{back-base, ij}=\bm{r}_{back,i} - \bm{r}_{base,j}$
$\bm{r}_{base-back, ij}=\bm{r}_{base,i} - \bm{r}_{back,j}$
\end{itemize}
\textbf{Relative angles} are defined using the above vectors, the base vector $\bm{b}$ and base normal $\bm{n}$ 
\vspace*{0.25cm}
\begin{itemize}
\item[] $\cos(\theta_1) = -\,\hat{\bm{b}}_i \cdot \hat{\bm{b}}_j$
\item[] $\cos(\theta_2) = -\,\hat{\bm{b}}_i \cdot \hat{\bm{r}}_{base, ij}$
\item[] $\cos(\theta_3) = \hat{\bm{b}}_j \cdot \hat{\bm{r}}_{base, ij}$
\item[] $\qquad\vdots\qquad\vdots\qquad\vdots$
\end{itemize}
\end{column}

\begin{column}{0.44\textwidth}
\vspace*{-0.25cm}
\begin{center}
\includegraphics[width=0.9\textwidth]{oxdna.jpg}
\textit{oxDNA1 vectors and angles}
\end{center}
\end{column}
\end{columns}

\end{frame}

\begin{frame}
\frametitle{Potential Forms}
\textbf{Elementary potentials} are used, which take distances or angles as arguments.\\[10pt]
\begin{itemize}
\setlength\itemsep{7pt}
\item FENE springs for backbone connectivity\\
$V_{FENE}(r,\epsilon,r^0,\Delta)=-\frac{\epsilon}{2}\ln\left(1-\frac{(r-r^0)^2}{\Delta^2}\right)$
\item Morse potential for stacking and hydrogen-bonding\\
$V_{Morse}(r,\epsilon,r^0,a)=\epsilon(1-\exp(-a(r-r^0)))^2$
\item Harmonic potential for cross-stacking and coaxial stacking\\
$V_{harm}(r,k,r^0)=\frac{k}{2}(r-r^0)^2$
\item Lennard-Jones potential for excluded volume\\
$V_{LJ}(r,\epsilon,\sigma)=4\epsilon\left(\left(\frac{\sigma}{r}\right)^{12} - \left(\frac{\sigma}{r}\right)^6\right)$
\item Quadratic terms for angular modulations\\
$V_{mod}(\theta,a,\theta^0)=1-a(\theta-\theta^0)^2$
\item Quadratic smoothing terms for truncation\\
$V_{smooth}(x,b,x^c)=b(x-x^c)^2$
\item Debye-H\"uckel potential for elextrostatics\\
$V_{DH}(r, \lambda)=\frac{q_{eff}}{4\pi\epsilon_0\epsilon_r} \exp(-r/\lambda)/r$
\end{itemize}

\end{frame}

\begin{frame}
\frametitle{Modulation Factors}
The above potentials are used directly or in angular and radial modulation factors $f_{1,\dots,6}$.
\scriptsize
\begin{flalign*}
  &f_{1}(r) = 
   \begin{cases}
  V_{Morse}(r, \epsilon, r^{0}, a) &\mbox{if } r^{low} < r < r^{high}, \\
  \epsilon V_{smooth}(r, b^{low}, r^{c,low}) &\mbox{if } r^{c, low} < r < r^{low}, \\
  \epsilon V_{smooth}(r, b^{high}, r^{c,high}) &\mbox{if } r^{high} < r < r^{c, high}, \\
  0 &\mbox{otherwise}
  \end{cases}\\
  &f_{2}(r) = 
  \begin{cases}
  V_{harm}(r, k, r^{0})-V_{harm}(r^{c}, k, r^{0}) &\mbox{if } r^{low} < r , r^{high}, \\
  kV_{smooth}(r , b^{low}, r^{c,low}) &\mbox{if } r^{c, low} < r < r^{low}, \\
  kV_{smooth}(r , b^{high}, r^{c,high}) &\mbox{if } r^{high} < r < r^{c, high}, \\
  0 &\mbox{otherwise}
  \end{cases}\\
  &f_{3}(r) = 
  \begin{cases}
  V_{LJ}(r, \epsilon, \sigma) &\mbox{if } r < r^{\star}, \\
  \epsilon V_{smooth}(r, b, r^{c}) &\mbox{if } r^{\star} < r < r^{c}, \\
  0 &\mbox{otherwise}
  \end{cases}\\
  &f_{4}(\theta) = 
  \begin{cases}
  V_{mod}(\theta, a, \theta^{0}) &\mbox{if } \theta^{0} - \Delta\theta^{\star} < \theta < \theta^{0} + \Delta\theta^{\star}, \\
  V_{smooth}(\theta, b, \theta^{0} - \Delta\theta^{c}) &\mbox{if } \theta^{0} - \Delta\theta^{c} < \theta < \theta^{0} - \Delta\theta^{\star}, \\
  V_{smooth}(\theta, b, \theta^{0} + \Delta\theta^{c}) &\mbox{if } \theta^{0} + \Delta\theta^{\star} < \theta < \theta^{0} + \Delta\theta^{c}, \\
  0 &\mbox{otherwise}
  \end{cases}\\
  &f_{5}(x) = 
  \begin{cases}
  1 &\mbox{if } x > 0, \\
  V_{mod}(x,a,0) &\mbox{if } x^{\star} < x < 0, \\
  V_{smooth}(x, b, x^c) &\mbox{if } x^{c} < x< x^{\star}, \\
  0 &\mbox{otherwise}
  \end{cases}\\
  &f_{6}(\theta) = 
  \begin{cases}
  V_{smooth}(\theta, b, \theta^c) &\mbox{if } \theta\ge\theta^{c}, \\
  0 &\mbox{otherwise}
  \end{cases}\\
\end{flalign*}

\end{frame}

\begin{frame}
\frametitle{Interactions}
The \textbf{oxDNA2 potential} consists of \textbf{1 bonded} and \textbf{6 pair interactions}.
\vspace*{0.25cm}
\begin{itemize}
\setlength\itemsep{5pt}
\small
\item \textbf{Backbone connectivity} (bonded): $V_{backbone} = V_{FENE}(.)$
\item \textbf{Excluded volume} (pair)\\
$V_{excv} = f_3(r_{back-back}, ..)+f_3(r_{back-base}, ..)+f_3(r_{base-back}, ..)+f_3(r_{base-base}, ..)$
\item \textbf{Stacking} (pair): $V_{stack} = f_1(\cdot)\times f_4(\cdot)\times f_4(\cdot)\times f_4(\cdot)\times f_5(\cdot)\times f_5(\cdot)$
\item \textbf{Hydrogen-bonding} (pair): $V_{HB} = f_1(\cdot)\times f_4(\cdot)\times f_4(\cdot)\times f_4(\cdot)\times f_4(\cdot)\times f_4(\cdot)$
\item \textbf{Cross-stacking} (pair)\\
$V_{x-stack} = f_2(\cdot)\times f_4(\cdot)\times f_4(\cdot)\times f_4(\cdot)\times\Big\{f_4(\cdot)+ f_4(\cdot)\Big\} \times \Big\{f_4(\cdot)+ f_4(\cdot)\Big\} \times\Big\{f_4(\cdot)+ f_4(\cdot)\Big\}$
\item \textbf{Coaxial stacking} (pair)\\
$V_{coaxial-stack} = f_2(\cdot)\times f_4(\cdot)\times \times\Big\{f_4(\cdot)+ f_6(\cdot)\Big\} \times \Big\{f_4(\cdot)+ f_4(\cdot)\Big\} \times\Big\{f_4(\cdot)+ f_4(\cdot)\Big\}$
\item \textbf{Electrostatic} (pair): $V_{elec} = V_{DH}(\cdot)$
\end{itemize}

\vspace*{0.25cm}
The complete potential contains sums over consecutive nucleotides on the same strand and all other pairs.
\begin{flalign*}
V = &\sum_{nearest\ neighbours}(V_{backbone} + V'_{excv} + V_{stack}) \\ 
+ &\sum_{other\ pairs}(V_{elec} + V_{HB} + V_{x-stack} + V_{coaxial-stack} + V_{elec})
\end{flalign*}

\end{frame}


\begin{frame}
\frametitle{Summary}
\vspace*{0.25cm}
\small
\begin{itemize}
\setlength\itemsep{5pt}
\item The oxDNA model uses a \textbf{top-down coarse-graining approach} with rather complex \textbf{bespoke interactions}.
\item The oxDNA potential comprises \textbf{one bonded interaction} and \textbf{six pair interactions}. As strands denature, there is no residual memory of other conformations as is often the case with 3+body interactions.
\item The \textbf{thermodynamic properties} of oxDNA are basically those of the \textbf{SantaLucia nearest-neighbour model}, thought to be an \textbf{exact empirical fit} experiments. 
\item \textbf{Uniquely among coarse-grained  models} at this level of detail, oxDNA is able to describe the \textbf{thermodynamics of duplex formation} and provide an \textbf{accurate average representation} of the structure and mechanics of \textbf{both single-stranded and double stranded DNA and its assemblies}.
\item oxDNA2 features \textbf{sequence-specific hydrogen-bonding and stacking interaction strengths}. But there is \textbf{no intrinsic sequence-specific curvature or elasticity} ($\Rightarrow$ oxDNA3).  
\end{itemize}
\vspace*{0.25cm}
[1] T. Ouldridge, A. Louis, and J. Doye, \href{https://doi.org/10.1063/1.3552946}{Structural, Mechanical, and Thermodynamic Properties of a Coarse-Grained DNA Model}, \textit{J. Chem. Phys.} \textbf{134}, 085101 (2011).\\[7pt]
[2] B. Snodin, et al., \href{https://doi.org/10.1063/1.4921957}{Introducing Improved Structural Properties and Salt Dependence into a Coarse-Grained Model of DNA}, \textit{J. Chem. Phys.}  \textbf{142}, 234901 (2015).

\end{frame}


\section{oxDNA Software}

\begin{frame}
\frametitle{LAMMPS Version}
\begin{itemize}
\item \textbf{L}arge-scale \textbf{A}tomic/\textbf{M}olecular \textbf{M}assively \textbf{P}arallel \textbf{S}imulator
\item Available from the LAMMPS website at \href{https://www.lammps.org}{https://www.lammps.org}\\
Distributed under GPL v2 by Sandia National Laboratories\\
Latest stable release 23rd June 2022 (initial release 1995)
\item Popular
\begin{itemize}
\item 405,000 downloads between September 2004 and June 2021
\item 1,400 forks on GitHub, ca. 100 direct contributors
\end{itemize}
\item \textbf{Versatile}
\begin{itemize}
\item Very \textbf{advanced molecular dynamics capabilities}
\item Code distributed over \textbf{91 standard and \texttt{USER} packages}
\item Supported on \textbf{CPU-, multicore- and GPU-architectures}, but not all packages offer all options
\item REPLICA: collection of multi-replica methods, e.g. parallel tempering
\item PLUMED: free energy library, enhanced sampling
\item COLVARS: collective variables library, advanced sampling methods like metadynamics, umbrella sampling, adaptive biasing force 
\end{itemize}
\item \textbf{Extendable}
\begin{itemize}
\item Object-oriented C++ class structure
\item Top-level classes that are visible everywhere in the code
\item Virtual parent classes derived from top-level classes
\item Extensive use of polymorphism
\end{itemize}
\end{itemize}


\end{frame}

\begin{frame}[fragile]
\frametitle{LAMMPS Version}
\small Building the LAMMPS version with standard make
\begin{itemize}
\item Requires C/C++ compiler that supports the C++11 standard
\item Change to \texttt{/src} in your LAMMPS directory
\item Load \texttt{ASPHERE}, \texttt{MOLECULE} and \texttt{\textcolor{red}{CG-DNA}} packages (minimal requirement)\\
\linespread{0.4}
\begin{lstlisting}
make yes-asphere yes-molecule yes-cg-dna
\end{lstlisting}
\item Check modules are loaded and clean

\begin{lstlisting}
make ps

Installed YES: package ASPHERE
Installed YES: package CG-DNA
Installed YES: package MOLECULE
\end{lstlisting}

\item Compile the serial and/or parallel version using the default \texttt{Makefiles} in \texttt{/src/MAKE}
\begin{lstlisting}
make [-j4] serial
make [-j4] mpi
\end{lstlisting}
\item More \texttt{Makefile} configurations are in \texttt{/src/MAKE/MACHINES}
\end{itemize}
\linespread{1.0}\vspace*{0.25cm}
[3] O. Henrich, Y. A. Guti\'errez Fosado, T. Curk, and T. E. Ouldridge, \href{https://doi.org/10.1140/epje/i2018-11669-8}{Coarse-Grained Simulation of DNA Using LAMMPS: An Implementation of the OxDNA Model and Its Applications}, \textit{Eur. Phys. J. E} \textbf{41}, (2018).\\[3pt]
[4] LAMMPS CG-DNA Documentation \href{https://docs.lammps.org/PDF/CG-DNA.pdf}{https://docs.lammps.org/PDF/CG-DNA.pdf}
\end{frame}

\begin{frame}
\frametitle{Standalone Version}

\begin{itemize}
\item \textbf{oxDNA} code includes oxDNA1, oxDNA2 and oxRNA 
\item Available from \href{https://github.com/lorenzo-rovigatti/oxDNA}{https://github.com/lorenzo-rovigatti/oxDNA}\\
Distributed under GPL v3\\
Latest stable release 3.4.2 (8th Sept 2022)
\item 11 forks on GitHub, half a dozen contributors
\item Very \textbf{advanced Monte Carlo capabilities} like \textbf{Virtual-Move Monte Carlo} (VMMC) 
\item Supported on \textbf{single-core CPU- and single GPU-architectures}
\item Very extensive suite of \textbf{oxDNA Analysis Tools (OAT)} of bespoke postprocessing and analysis scripts 
\end{itemize}

\end{frame}

\begin{frame}[fragile]
\frametitle{Standalone Version}
Building the serial standalone version for CPU-architectures with standard CMake

\begin{itemize}
\item Requires

\begin{itemize}
\item C/C++ compiler that supports the C++14 standard
\item CMake version $\ge 3.5$
\item optionally CUDA toolkit version $\ge 10$
\end{itemize}
\item Change to the oxDNA top-level directory
\linespread{0.4}
\begin{lstlisting}
cd oxDNA
\end{lstlisting}
\item Create a build directory and change to it
\begin{lstlisting}
mkdir build
cd build
\end{lstlisting}
\item Create Makefiles, specify additonal options
\begin{lstlisting}
cmake ..
\end{lstlisting}
\item Compile the serial version
\begin{lstlisting}
make [-j4]
\end{lstlisting}
\end{itemize}

[5] oxDNA Documentation \href{https://lorenzo-rovigatti.github.io/oxDNA}{https://lorenzo-rovigatti.github.io/oxDNA}\\[3pt]
[6] oxDNA Website \href{https://dna.physics.ox.ac.uk}{https://dna.physics.ox.ac.uk}

\end{frame}

\begin{frame}
\frametitle{tacoxDNA Tools and Converters}

Available
\begin{itemize}
\item as webserver at \href{http://tacoxdna.sissa.it}{http://tacoxdna.sissa.it}
\item as standalone Python code from \href{https://github.com/lorenzo-rovigatti/tacoxDNA}{https://github.com/lorenzo-rovigatti/tacoxDNA}
\end{itemize}

\vspace*{0.5cm}
\includegraphics[width=0.45\textwidth]{tacoxDNA_pdb.jpg}
\includegraphics[width=0.45\textwidth]{tacoxDNA_trefoil.jpg}\\
\vspace*{0.5cm}

[7] A. Suma, et al., \href{https://doi.org/10.1002/jcc.26029}{TacoxDNA: A User-Friendly Web Server for Simulations of Complex DNA Structures, from Single Strands to Origami}, \textit{J. Comput. Chem.} \textbf{40}, 2586 (2019).

\end{frame}
\begin{frame}

\frametitle{tacoxDNA Tools and Converters}

\begin{columns}
\begin{column}{0.55\textwidth}
\vspace*{0.25cm}\\
A variety of format conversions is supported. The native oxDNA format can also be used as intermediate.
\vspace*{0.25cm}
\begin{itemize}
\setlength\itemsep{7pt}
\item LAMMPS format $\Leftrightarrow$ native oxDNA format
\item PDB format $\Leftrightarrow$ native oxDNA format
\item XYZ format $\Rightarrow$ native oxDNA format
\item cadnano $\Rightarrow$ native oxDNA format
\item CanDo $\Rightarrow$ native oxDNA format
\item Tiamat $\Rightarrow$ native oxDNA format
\end{itemize}
\end{column}

\begin{column}{0.45\textwidth}
\begin{center}
\includegraphics[width=0.90\textwidth]{tacoxDNA_schematic.jpg}
\end{center}
\end{column}
\end{columns}
\vspace*{0.25cm}
[7] A. Suma, et al., \href{https://doi.org/10.1002/jcc.26029}{TacoxDNA: A User-Friendly Web Server for Simulations of Complex DNA Structures, from Single Strands to Origami}, \textit{J. Comput. Chem.} \textbf{40}, 2586 (2019).

\end{frame}

\begin{frame}
\frametitle{oxViewer Visualisation and Manipulation Toolkit}
\small
\begin{columns}
\begin{column}{0.5\textwidth}
\end{column}
\begin{column}{0.5\textwidth}
\begin{center}
\vspace*{-0.25cm}
\includegraphics[width=\textwidth]{oxviewer.png}
\end{center}
\end{column}
\end{columns}

[8] E. Poppleton, et al., \href{https://doi.org/10.1093/nar/gkaa417}{Design, Optimization and Analysis of Large DNA and RNA Nanostructures through Interactive Visualization, Editing and Molecular Simulation}, \textit{Nucleic Acids Res.} \textbf{48}, e72 (2020).\\[3pt]

[9] J. Bohlin, et al., \href{https://doi.org/10.1038/s41596-022-00688-5}{Design and Simulation of DNA, RNA and Hybrid Protein–Nucleic Acid Nanostructures with OxView}, \textit{Nat. Protoc.} \textbf{17}, 1762 (2022).


\end{frame}

\section{Practical Exercises}

\begin{frame}
\frametitle{Melting}
BlaBlaBlaBlub

\end{frame}

\end{document}
