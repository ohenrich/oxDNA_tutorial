% General settings
\documentclass[slidestop,compress,9pt]{beamer}
\usetheme{AnnArbor}
\usepackage[UKenglish]{babel}
\usepackage[UKenglish]{isodate}
\usepackage[utf8]{inputenc}
\usepackage{hyperref}
\definecolor{links}{HTML}{2A1B81}
\hypersetup{colorlinks=true,allcolors=links}

% Code listings and pseudocode
\usepackage{listings}
\usepackage{algpseudocode}
\usepackage{algorithm}

% Mathematical formulas
\usepackage{amsmath}
\usepackage{mathtools}
\usepackage{bm}
\setcounter{MaxMatrixCols}{20}

% Graphics and figures
\usepackage{graphicx}
\graphicspath{{figures/}}
\setbeamertemplate{caption}[numbered]
\usepackage{fancybox}
\usepackage{pgfplots}

% Tikz
\usepackage{tikz}
\usepackage{tikz-3dplot}
\usetikzlibrary{shapes.geometric, arrows}

% For redefinition of texttt
%\let\textttorig\texttt
\usepackage[scaled=1.0]{couriers}

\setbeamertemplate{frametitle}{\thesection.~\insertsection\hspace*{0.45cm}\insertframetitle}

% Numbered sections
\setbeamertemplate{section in toc}[sections numbered]
\setbeamertemplate{subsection in toc}[subsections numbered]
\setbeamertemplate{subsubsection in toc}[subsubsections numbered]

% Listings
\definecolor{cident}{rgb}{0.0,0.0,0.0}
\definecolor{ckeyw}{rgb}{0,0,0.8}
\definecolor{ccomm}{rgb}{0,0.8,0}
\definecolor{cstr}{rgb}{0.8,0,0}
\definecolor{myyellow}{rgb}{0.99,0.76, 0.0}
\definecolor{mymagenta}{rgb}{1.0, 0.0, 1.0}

\lstset{language=Python,
  basicstyle=\small\ttfamily,
  keywordstyle=\color{ckeyw}\bfseries,
  identifierstyle=\color{cident}\bfseries,
  commentstyle=\color{ccomm},
  stringstyle=\color{cstr},
  showstringspaces=false,
  breaklines=true,
  breakatwhitespace=true,
  tabsize=2,
  mathescape = false,
  columns=flexible,
  escapeinside={<@}{@>}
%   numbers=left,
%   stepnumber=1,
%   firstnumber=1,
%   numberfirstline=true,
  }

\title[The oxDNA Coarse-Grained Model of DNA and RNA]{\textbf{The oxDNA Coarse-Grained Model of DNA}}
\subtitle{An Introduction to the Model and Software Framework}
\author[Oliver Henrich]{Oliver Henrich \\ \small Email: \href{mailto:oliver.henrich@strath.ac.uk}{oliver.henrich@strath.ac.uk}}
\institute[U Strathclyde, Glasgow, UK]{Department of Physics, University of Strathclyde, Glasgow, UK}
\date{30th April 2024}


\begin{document}

%\renewcommand<>{\texttt}[1]{%
%  \only#2{\textttorig{#1}}%
%}

\renewcommand<>{\texttt}{\only#1{\beameroriginal{\texttt}}}

% ==============================================================
% --- Welcome frame
% ==============================================================
\begin{frame}[plain]
\maketitle
\end{frame}

% ==============================================================
% --- TOC
% ==============================================================

\begin{frame}
\Large Outline
\normalsize
\tableofcontents
\begin{center}
GitHub tutorial repo at \href{https://github.com/ohenrich/oxDNA\_tutorial}{https://github.com/ohenrich/oxDNA\_tutorial}
\end{center}
\vspace*{3cm}
\end{frame}

% ==============================================================
% --- Content
% ==============================================================

\section{oxDNA Model}

\begin{frame}
\frametitle{DNA Facts}

\begin{itemize}
\item Double-stranded DNA is $\approx 2 nm$ wide, but $2m$ long in one cell (or $100 AU = 0.58$ light days in all cells)
\item Human genome contains $3\times10^{9}$ base pairs (bps)
\item DNA loop around a nucleosome core particle contains 147 bps
\item Smallest loop in chromatin fibre consists of $5\times10^4$ bps
\item Atomistic simulation of DNA can model 3,000 bps (probably less) and typically resolve times on the $\mu$s-scale  
\item \textbf{Coarse-grained models} target much \textbf{larger time and length scales} in the range of ms and Mbps 
\end{itemize}

\vspace*{-0.25cm}
\begin{center}
\includegraphics[width=0.54\textwidth]{fromDNAtoChromatin.png}
\includegraphics[width=0.45\textwidth]{A-B-Z-DNA.png}\\
\textit{From DNA to chromatin}\hspace{3.2cm}\textit{A-DNA}\hspace{0.9cm} \textit{B-DNA}\hspace{0.9cm}\textit{Z-DNA}
\end{center}

\end{frame}

\begin{frame}
\frametitle{Overview}

\begin{itemize}
\item \textbf{Each nucleotide} is described as \textbf{rigid body}
\item \textbf{3 interaction} sites for backbone, stacking and hydrogen-bonding
\item \textbf{7 effective interactions} between nucleotides
\begin{itemize}
\item Bonded interaction for backbone connectivity
\item 6 pair interactions for excluded volume, stacking, cross-stacking, coaxial stacking, hydrogen-bonding and electrostatic interaction
\end{itemize}
\item \textbf{oxDNA: 13 DOF per nucleotide}\\
3 positions, 3 translational momenta, 3 angular momenta, 1 unit quaternion (4 components)
\item \textbf{Atomistic simulation} (pyrimidine base plus sugar-phosphate group):\\
around \textbf{200 DOF per nucleotide}\\
34 atoms per nucleotide, each with 3 positions and 3 momenta
\end{itemize}

\vspace*{-0.5cm}

\begin{columns}
\begin{column}{0.4\textwidth}
\begin{center}
\includegraphics[width=0.9\textwidth]{nucleotide_duplex_b.pdf}
\end{center}
\end{column}

\begin{column}{0.45\textwidth}
\begin{center}
\vspace*{0.5cm}
\begin{itemize}
\setlength\itemsep{7pt}
\item[] \textit{(a) oxDNA nucleotide}
\item[] \textit{(b) Duplex}
\item[] \textit{(c) Interactions}
\end{itemize}
\end{center}
\end{column}
\end{columns}

\end{frame}

\begin{frame}
\frametitle{Nucleotide Geometry}

\begin{columns}
\begin{column}{0.56\textwidth}
\begin{itemize}
\setlength\itemsep{20pt}
\item Each nucleotide has a centre of mass (COM) $\bm{r}_{COM}$, a base vector $\bm{b}$, base normal $\bm{n}$ and a third vector $\bm{y}=\bm{n}\times\bm{b}$
\item The \textbf{backbone interaction} site is at\\
$\bm{r}_{back}=\bm{r}_{COM} - 0.4\,\bm{b}$ \hspace{1.7cm}(oxDNA1)\\
$\bm{r}_{back}=\bm{r}_{COM} - 0.34\, \bm{b} + 0.3408\,\bm{y}$ (oxDNA2)
\item The \textbf{stacking interaction} site is at\\
$\bm{r}_{stack}=\bm{r}_{COM} + 0.34\, \bm{b}$
\item The \textbf{hydrogen-bonding interaction} site is at\\
$\bm{r}_{base}=\bm{r}_{COM} + 0.4\, \bm{b}$
\end{itemize}
\end{column}

\begin{column}{0.44\textwidth}
\vspace*{-0.5cm}
\begin{center}
\includegraphics[width=0.85\textwidth]{oxdna_oxdna2.png}\\
\end{center}
\textit{(a) oxDNA1 and oxDNA2 nucleotides: the base vector $\bm{b}$ is horizontally oriented from left to right, whereas the base normal $\bm{n}$ points away from the observer.}\\[3pt]
\textit{(b) The angled backbone interaction sites leads to the correct geometry with major and minor grooves.} 
\end{column}
\end{columns}

\end{frame}

\begin{frame}
\frametitle{Angles and Vectors}

\begin{columns}
\begin{column}{0.56\textwidth}
\textbf{Relative distance vectors} are defined between two nucleotides $i$ and $j$
\begin{itemize}
\setlength\itemsep{3pt}
\item backbone interaction sites\\
$\bm{r}_{back, ij}=\bm{r}_{back,i} - \bm{r}_{back,j}$
\item stacking interaction sites\\
$\bm{r}_{stack, ij}=\bm{r}_{stack,i} - \bm{r}_{stack,j}$
\item hydrogen-bonding interaction sites\\
$\bm{r}_{base, ij}=\bm{r}_{base,i} - \bm{r}_{base,j}$
\item mixed sites\\
$\bm{r}_{back-base, ij}=\bm{r}_{back,i} - \bm{r}_{base,j}$
$\bm{r}_{base-back, ij}=\bm{r}_{base,i} - \bm{r}_{back,j}$
\end{itemize}
\textbf{Relative angles} are defined using the above vectors, the base vector $\bm{b}$ and base normal $\bm{n}$ 
\vspace*{0.25cm}
\begin{itemize}
\item[] $\cos(\theta_1) = -\,\hat{\bm{b}}_i \cdot \hat{\bm{b}}_j$
\item[] $\cos(\theta_2) = -\,\hat{\bm{b}}_i \cdot \hat{\bm{r}}_{base, ij}$
\item[] $\cos(\theta_3) = \hat{\bm{b}}_j \cdot \hat{\bm{r}}_{base, ij}$
\item[] $\qquad\vdots\qquad\vdots\qquad\vdots$
\end{itemize}
\end{column}

\begin{column}{0.44\textwidth}
\vspace*{-0.25cm}
\begin{center}
\includegraphics[width=0.9\textwidth]{oxdna.jpg}
\textit{oxDNA1 vectors and angles}
\end{center}
\end{column}
\end{columns}

\end{frame}

\begin{frame}
\frametitle{Potential Forms}
\textbf{Elementary potentials} are used, which take distances or angles as arguments.\\[10pt]
\begin{itemize}
\setlength\itemsep{7pt}
\item FENE springs for backbone connectivity\\
$V_{FENE}(r,\epsilon,r^0,\Delta)=-\frac{\epsilon}{2}\ln\left(1-\frac{(r-r^0)^2}{\Delta^2}\right)$
\item Morse potential for stacking and hydrogen-bonding\\
$V_{Morse}(r,\epsilon,r^0,a)=\epsilon(1-\exp(-a(r-r^0)))^2$
\item Harmonic potential for cross-stacking and coaxial-stacking\\
$V_{harm}(r,k,r^0)=\frac{k}{2}(r-r^0)^2$
\item Lennard-Jones potential for excluded volume\\
$V_{LJ}(r,\epsilon,\sigma)=4\epsilon\left(\left(\frac{\sigma}{r}\right)^{12} - \left(\frac{\sigma}{r}\right)^6\right)$
\item Quadratic terms for angular modulations\\
$V_{mod}(\theta,a,\theta^0)=1-a(\theta-\theta^0)^2$
\item Quadratic smoothing terms for truncation\\
$V_{smooth}(x,b,x^c)=b(x-x^c)^2$
\item Debye-H\"uckel potential for electrostatics\\
$V_{DH}(r, \lambda)=\frac{q_{eff}}{4\pi\epsilon_0\epsilon_r} \exp(-r/\lambda)/r$
\end{itemize}

\end{frame}

\begin{frame}
\frametitle{Modulation Factors}
The above potentials are used directly or in angular and radial modulation factors $f_{1,\dots,6}$.
\scriptsize
\begin{flalign*}
  &f_{1}(r) = 
   \begin{cases}
  V_{Morse}(r, \epsilon, r^{0}, a) &\mbox{if } r^{low} < r < r^{high}, \\
  \epsilon V_{smooth}(r, b^{low}, r^{c,low}) &\mbox{if } r^{c, low} < r < r^{low}, \\
  \epsilon V_{smooth}(r, b^{high}, r^{c,high}) &\mbox{if } r^{high} < r < r^{c, high}, \\
  0 &\mbox{otherwise}
  \end{cases}\\
  &f_{2}(r) = 
  \begin{cases}
  V_{harm}(r, k, r^{0})-V_{harm}(r^{c}, k, r^{0}) &\mbox{if } r^{low} < r , r^{high}, \\
  kV_{smooth}(r , b^{low}, r^{c,low}) &\mbox{if } r^{c, low} < r < r^{low}, \\
  kV_{smooth}(r , b^{high}, r^{c,high}) &\mbox{if } r^{high} < r < r^{c, high}, \\
  0 &\mbox{otherwise}
  \end{cases}\\
  &f_{3}(r) = 
  \begin{cases}
  V_{LJ}(r, \epsilon, \sigma) &\mbox{if } r < r^{\star}, \\
  \epsilon V_{smooth}(r, b, r^{c}) &\mbox{if } r^{\star} < r < r^{c}, \\
  0 &\mbox{otherwise}
  \end{cases}\\
  &f_{4}(\theta) = 
  \begin{cases}
  V_{mod}(\theta, a, \theta^{0}) &\mbox{if } \theta^{0} - \Delta\theta^{\star} < \theta < \theta^{0} + \Delta\theta^{\star}, \\
  V_{smooth}(\theta, b, \theta^{0} - \Delta\theta^{c}) &\mbox{if } \theta^{0} - \Delta\theta^{c} < \theta < \theta^{0} - \Delta\theta^{\star}, \\
  V_{smooth}(\theta, b, \theta^{0} + \Delta\theta^{c}) &\mbox{if } \theta^{0} + \Delta\theta^{\star} < \theta < \theta^{0} + \Delta\theta^{c}, \\
  0 &\mbox{otherwise}
  \end{cases}\\
  &f_{5}(x) = 
  \begin{cases}
  1 &\mbox{if } x > 0, \\
  V_{mod}(x,a,0) &\mbox{if } x^{\star} < x < 0, \\
  V_{smooth}(x, b, x^c) &\mbox{if } x^{c} < x< x^{\star}, \\
  0 &\mbox{otherwise}
  \end{cases}\\
  &f_{6}(\theta) = 
  \begin{cases}
  V_{smooth}(\theta, b, \theta^c) &\mbox{if } \theta\ge\theta^{c}, \\
  0 &\mbox{otherwise}
  \end{cases}\\
\end{flalign*}

\end{frame}

\begin{frame}
\frametitle{Interactions}
The \textbf{oxDNA2 potential} consists of \textbf{1 bonded} and \textbf{6 pair interactions}.
\vspace*{0.25cm}
\begin{itemize}
\setlength\itemsep{5pt}
\small
\item \textbf{Backbone connectivity} (bonded): $V_{backbone} = V_{FENE}(.)$
\item \textbf{Excluded volume} (pair)\\
$V_{excv} = f_3(r_{back-back}, ..)+f_3(r_{back-base}, ..)+f_3(r_{base-back}, ..)+f_3(r_{base-base}, ..)$
\item \textbf{Stacking} (pair): $V_{stack} = f_1(\cdot)\times f_4(\cdot)\times f_4(\cdot)\times f_4(\cdot)\times f_5(\cdot)\times f_5(\cdot)$
\item \textbf{Hydrogen-bonding} (pair): $V_{HB} = f_1(\cdot)\times f_4(\cdot)\times f_4(\cdot)\times f_4(\cdot)\times f_4(\cdot)\times f_4(\cdot)$
\item \textbf{Cross-stacking} (pair)\\
$V_{x-stack} = f_2(\cdot)\times f_4(\cdot)\times f_4(\cdot)\times f_4(\cdot)\times\Big\{f_4(\cdot)+ f_4(\cdot)\Big\} \times \Big\{f_4(\cdot)+ f_4(\cdot)\Big\} \times\Big\{f_4(\cdot)+ f_4(\cdot)\Big\}$
\item \textbf{Coaxial stacking} (pair)\\
$V_{coaxial-stack} = f_2(\cdot)\times f_4(\cdot)\times \times\Big\{f_4(\cdot)+ f_6(\cdot)\Big\} \times \Big\{f_4(\cdot)+ f_4(\cdot)\Big\} \times\Big\{f_4(\cdot)+ f_4(\cdot)\Big\}$
\item \textbf{Electrostatic} (pair): $V_{elec} = V_{DH}(\cdot)$
\end{itemize}

\vspace*{0.25cm}
The complete potential contains sums over consecutive nucleotides on the same strand and all other pairs.
\begin{flalign*}
V = &\sum_{nearest\ neighbours}(V_{backbone} + V'_{excv} + V_{stack}) \\ 
+ &\sum_{other\ pairs}(V_{excv} + V_{HB} + V_{x-stack} + V_{coaxial-stack} + V_{elec})
\end{flalign*}

\end{frame}


\begin{frame}
\frametitle{Summary}
\vspace*{0.25cm}
\small
\begin{itemize}
\setlength\itemsep{5pt}
\item The oxDNA model uses a \textbf{top-down coarse-graining approach} with rather complex \textbf{bespoke interactions}.
\item The oxDNA potential comprises \textbf{one bonded interaction} and \textbf{six pair interactions}. As strands denature, there is no residual memory of other conformations as is often the case with 3+body interactions.
\item The \textbf{thermodynamic properties} of oxDNA are basically those of the \textbf{SantaLucia nearest-neighbour model}, thought to be an \textbf{exact empirical fit} experiments. 
\item \textbf{Uniquely among coarse-grained  models} at this level of detail, oxDNA is able to describe the \textbf{thermodynamics of duplex formation} and provide an \textbf{accurate average representation} of the structure and mechanics of \textbf{both single-stranded and double stranded DNA and its assemblies}.
\item oxDNA2 features \textbf{sequence-specific hydrogen-bonding and stacking interaction strengths}. But there is \textbf{no intrinsic sequence-specific curvature or elasticity} ($\Rightarrow$ oxDNA3).  
\end{itemize}
\vspace*{0.2cm}
[1] T. Ouldridge, A. Louis, and J. Doye, \href{https://doi.org/10.1063/1.3552946}{Structural, Mechanical, and Thermodynamic Properties of a Coarse-Grained DNA Model}, \textit{J. Chem. Phys.} \textbf{134}, 085101 (2011).\\[3pt]
[2] B. Snodin, et al., \href{https://doi.org/10.1063/1.4921957}{Introducing Improved Structural Properties and Salt Dependence into a Coarse-Grained Model of DNA}, \textit{J. Chem. Phys.}  \textbf{142}, 234901 (2015).\\[3pt]
[3] A. Sengar, et al., \href{https://doi.org/10.3389/fmolb.2021.693710}{A primer on the oxDNA model of DNA: When to use it, how to simulate it and how to interpret the results}, \textit{Front. Mol. Biosci.}  \textbf{8}, 693710 (2021).
\end{frame}


\section{oxDNA Software}

\begin{frame}
\frametitle{Overview}

We will use the following software\\[10pt]

\begin{itemize}
\setlength\itemsep{10pt}
\item LAMMPS version of oxDNA\\
The implementation of oxDNA1, oxDNA2 and oxRNA2 in the LAMMPS code via the \texttt{CG-DNA} package 
\item Standalone version of oxDNA\\
The original re-implementation of oxDNA1, which was extended to oxDNA2 and oxRNA2
\item tacoxDNA\\
A suite of tools and converters
\item oxView\\
A visualisation and data manipulation toolkit
\end{itemize}

\end{frame}

\begin{frame}
\frametitle{LAMMPS Version}
\begin{itemize}
\item \textbf{L}arge-scale \textbf{A}tomic/\textbf{M}olecular \textbf{M}assively \textbf{P}arallel \textbf{S}imulator
\item Available from the LAMMPS website at \href{https://www.lammps.org}{https://www.lammps.org}\\
Distributed under GPL v2 by Sandia National Laboratories\\
Latest stable release 2nd August 2023, updated 2nd March 2024 (initial release 1995)
\item Popular
\begin{itemize}
\item 405,000 downloads between September 2004 and June 2021
\item 1,600 forks on GitHub, ca. 100 direct contributors
\end{itemize}
\item \textbf{Versatile}
\begin{itemize}
\item Very \textbf{advanced molecular dynamics capabilities}
\item Code distributed over \textbf{ca. 100 standard and \texttt{USER} packages}
\item Supported on \textbf{CPU-, multicore- and GPU-architectures}, but not all packages offer all options
\item REPLICA: collection of multi-replica methods, e.g. parallel tempering
\item PLUMED: free energy library, enhanced sampling
\item COLVARS: collective variables library, advanced sampling methods like metadynamics, umbrella sampling, adaptive biasing force 
\end{itemize}
\item \textbf{Extendable}
\begin{itemize}
\item Object-oriented C++ class structure
\item Top-level classes that are visible everywhere in the code
\item Virtual parent classes derived from top-level classes
\item Extensive use of polymorphism
\end{itemize}
\end{itemize}


\end{frame}

\begin{frame}[fragile]
\frametitle{LAMMPS Version}
\small 
Building the LAMMPS version with standard make
\begin{itemize}
\item Requires C/C++ compiler that supports the C++11 standard
\item Change to \texttt{/src} in your LAMMPS directory
\item Load \texttt{ASPHERE}, \texttt{MOLECULE} and \texttt{\textcolor{red}{CG-DNA}} packages (minimal requirement)\\
\linespread{0.4}
\begin{lstlisting}
make yes-asphere yes-molecule yes-cg-dna
\end{lstlisting}
\item Check modules are loaded and clean

\begin{lstlisting}
make ps

Installed YES: package ASPHERE
Installed YES: package CG-DNA
Installed YES: package MOLECULE
\end{lstlisting}

\item Compile the serial and/or parallel version using the default \texttt{Makefiles} in \texttt{/src/MAKE}
\begin{lstlisting}
make [-j4] serial
make [-j4] mpi
\end{lstlisting}
\item More \texttt{Makefile} configurations are in \texttt{/src/MAKE/MACHINES}
\end{itemize}
\linespread{1.0}\vspace*{0.25cm}
[4] O. Henrich, Y. A. Guti\'errez Fosado, T. Curk, and T. E. Ouldridge, \href{https://doi.org/10.1140/epje/i2018-11669-8}{Coarse-Grained Simulation of DNA Using LAMMPS: An Implementation of the OxDNA Model and Its Applications}, \textit{Eur. Phys. J. E} \textbf{41}, (2018).\\[3pt]
[5] LAMMPS CG-DNA Documentation \href{https://docs.lammps.org/PDF/CG-DNA.pdf}{https://docs.lammps.org/PDF/CG-DNA.pdf}
\end{frame}

\begin{frame}
\frametitle{Standalone Version}

\begin{itemize}
\setlength\itemsep{10pt}
\item \textbf{oxDNA} code includes oxDNA1, oxDNA2 and oxRNA 
\item Available from \href{https://github.com/lorenzo-rovigatti/oxDNA}{https://github.com/lorenzo-rovigatti/oxDNA}\\
Distributed under GPL v3\\
Latest stable release 3.6.1 (12th March 2024)
\item 26 forks on GitHub, half a dozen contributors
\item Very \textbf{advanced Monte Carlo capabilities} like \textbf{Virtual-Move Monte Carlo} (VMMC) 
\item Supported on \textbf{single-core CPU- and single GPU-architectures}
\item Very extensive suite of \textbf{oxDNA Analysis Tools (OAT)} of bespoke postprocessing and analysis scripts 
\end{itemize}

\end{frame}

\begin{frame}[fragile]
\frametitle{Standalone Version}
Building the serial standalone version for CPU-architectures with standard CMake

\begin{itemize}
\item Requires

\begin{itemize}
\item C/C++ compiler that supports the C++14 standard
\item CMake version $\ge 3.5$
\item optionally CUDA toolkit version $\ge 10$
\end{itemize}
\item Change to the oxDNA top-level directory
\linespread{0.4}
\begin{lstlisting}
cd oxDNA
\end{lstlisting}
\item Create a build directory and change to it
\begin{lstlisting}
mkdir build
cd build
\end{lstlisting}
\item Create Makefiles, specify additonal options
\begin{lstlisting}
cmake ..
\end{lstlisting}
\item Compile the serial version
\begin{lstlisting}
make [-j4]
\end{lstlisting}
\end{itemize}

[6] oxDNA Documentation \href{https://lorenzo-rovigatti.github.io/oxDNA}{https://lorenzo-rovigatti.github.io/oxDNA}\\[3pt]
[7] oxDNA Website \href{https://dna.physics.ox.ac.uk}{https://dna.physics.ox.ac.uk}

\end{frame}

\begin{frame}[fragile]
\frametitle{Standalone Version}

Build with oxDNA Analysis Tools (OAT)\\[7pt]

\begin{itemize}
\item Create Makefiles with specific options
\begin{lstlisting}
cmake .. -DPython=ON -DOxpySystemInstall=On
\end{lstlisting}
\item It might be necessary to specify explicit paths.
\begin{lstlisting}
cmake .. -DPython=ON -DOxpySystemInstall=On 
            -DPYTHON_INCLUDE_DIRS=/path/to/python/include/dir
            -DPYTHON_EXECUTABLE=/path/to/python/binary
\end{lstlisting}
\item Compile and install
\begin{lstlisting}
make [-j4]
make install
\end{lstlisting}
\end{itemize}

\textbf{Note}: A full installation of Anaconda3 is highly recommended!\\[7pt]

More information regarding installation and known issues is available in the\\[3pt] 
[6] oxDNA documentation \href{https://lorenzo-rovigatti.github.io/oxDNA/install.html}{https://lorenzo-rovigatti.github.io/oxDNA/install.html}

\end{frame}

\begin{frame}
\frametitle{tacoxDNA Tools and Converters}

Available
\begin{itemize}
\item as webserver at \href{http://tacoxdna.sissa.it}{http://tacoxdna.sissa.it}
\item as standalone Python code from \href{https://github.com/lorenzo-rovigatti/tacoxDNA}{https://github.com/lorenzo-rovigatti/tacoxDNA}
\end{itemize}

\vspace*{0.5cm}
\includegraphics[width=0.45\textwidth]{tacoxDNA_pdb.jpg}
\includegraphics[width=0.45\textwidth]{tacoxDNA_trefoil.jpg}\\
\vspace*{0.5cm}

[8] A. Suma, et al., \href{https://doi.org/10.1002/jcc.26029}{TacoxDNA: A User-Friendly Web Server for Simulations of Complex DNA Structures, from Single Strands to Origami}, \textit{J. Comput. Chem.} \textbf{40}, 2586 (2019).

\end{frame}
\begin{frame}

\frametitle{tacoxDNA Tools and Converters}

\begin{columns}
\begin{column}{0.55\textwidth}
\vspace*{0.25cm}\\
A variety of format conversions is supported. The native oxDNA format can also be used as intermediate.
\vspace*{0.25cm}
\begin{itemize}
\setlength\itemsep{7pt}
\item LAMMPS format $\Leftrightarrow$ native oxDNA format
\item PDB format $\Leftrightarrow$ native oxDNA format
\item XYZ format $\Rightarrow$ native oxDNA format
\item cadnano $\Rightarrow$ native oxDNA format
\item CanDo $\Rightarrow$ native oxDNA format
\item Tiamat $\Rightarrow$ native oxDNA format
\end{itemize}
\end{column}

\begin{column}{0.45\textwidth}
\begin{center}
\includegraphics[width=0.90\textwidth]{tacoxDNA_schematic.jpg}
\end{center}
\end{column}
\end{columns}
\vspace*{0.25cm}
[8] A. Suma, et al., \href{https://doi.org/10.1002/jcc.26029}{TacoxDNA: A User-Friendly Web Server for Simulations of Complex DNA Structures, from Single Strands to Origami}, \textit{J. Comput. Chem.} \textbf{40}, 2586 (2019).

\end{frame}

\begin{frame}[fragile]
\frametitle{oxView Visualisation and Manipulation Toolkit}
\small

\begin{columns}

\begin{column}{0.65\textwidth}
\textbf{oxView} is a webbrowser-based visualiser
\begin{itemize}
\item Can load structures with over 1 million nucleotides
\item Create videos from simulation trajectories
\item Allow users to perform basic edits to DNA and RNA designs
\end{itemize}
\vspace*{0.1cm}
Available from \href{https://github.com/sulcgroup/oxdna-viewer}{https://github.com/sulcgroup/oxdna-viewer}  
\begin{itemize}
\item Navigate down to the \texttt{README.md} and click \href{https://sulcgroup.github.io/oxdna-viewer}{Try it!}\\or
\item Load \href{https://sulcgroup.github.io/oxdna-viewer}{https://sulcgroup.github.io/oxdna-viewer}\\or
\item Run locally on \texttt{localhost:8000} by starting a python server in your oxView directory
\begin{lstlisting}
python -m http.server 8000
\end{lstlisting}
and open \href{http://localhost:8000}{http://localhost:8000} in your browser
\end{itemize}

\end{column}
\begin{column}{0.35\textwidth}
\begin{center}
\vspace*{-0.35cm}
\includegraphics[width=\textwidth]{oxviewer.png}
\end{center}
\end{column}
\end{columns}

\vspace*{0.25cm}

[9] E. Poppleton, et al., \href{https://doi.org/10.1093/nar/gkaa417}{Design, Optimization and Analysis of Large DNA and RNA Nanostructures through Interactive Visualization, Editing and Molecular Simulation}, \textit{Nucleic Acids Res.} \textbf{48}, e72 (2020).\\[3pt]

[10] J. Bohlin, et al., \href{https://doi.org/10.1038/s41596-022-00688-5}{Design and Simulation of DNA, RNA and Hybrid Protein–Nucleic Acid Nanostructures with OxView}, \textit{Nat. Protoc.} \textbf{17}, 1762 (2022).


\end{frame}

\section{Practical Exercises}

\begin{frame}[fragile]
\frametitle{(0) Compiling the Source Codes and Download}
\small

\begin{itemize}
\item Clone the LAMMPS code from GitHub and compile (C/C++11 compiler, MPI)
\begin{lstlisting}
git clone https://github.com/lammps/lammps.git
cd /lammps/src
make yes-asphere yes-molecule yes-cg-dna
make ps
make [-j4] mpi
\end{lstlisting}
\item Clone the oxDNA standalone code from GitHub and compile (C/C++14 compiler, CMake, Python $\ge$ 3.9, but Anaconda3 recommended)
\begin{lstlisting}
git clone https://github.com/lorenzo-rovigatti/oxDNA.git
cd /oxDNA
mkdir build
cd build
cmake .. -DPython=ON [-DOxpySystemInstall=On]
make [-j4]
make install
\end{lstlisting}
\item Clone tacoxDNA (Python3) and oxView (browser, Chrome recommended)
\begin{lstlisting}
git clone https://github.com/lorenzo-rovigatti/tacoxDNA.git
git clone https://github.com/sulcgroup/oxdna-viewer.git
\end{lstlisting}
This is only necessary if you want to use these tools in a standalone fashion, e.g. because you cannot connect to the internet.
\end{itemize}

\end{frame}


\begin{frame}[fragile]
\frametitle{(1) Creating an Initial Configuration}
\small

\begin{itemize}
\item Clone the oxDNA tutorial directory from GitHub.
\begin{lstlisting}
git clone https://github.com/ohenrich/oxDNA_tutorial.git 
\end{lstlisting}

\item Navigate to the first exercise.
\begin{lstlisting}
cd exercises/1_initial_config
\end{lstlisting}

\item Inspect the file \texttt{sequence.txt}.
\begin{lstlisting}
vi sequence.txt
DOUBLE AAAAAACGCGAAA...
\end{lstlisting}
The keyword \texttt{DOUBLE} indicates that we will create double-stranded DNA. Omitting the keyword produces a single-stranded configuration.

\item Check syntax of configuration generator.
\linespread{0.4}
\begin{lstlisting}
python generate-sa.py

Usage: generate-sa.py <box size> <file with sequences>
\end{lstlisting}

\item Generate an initial configuration in a box size 100.
\begin{lstlisting}
python generate-sa.py 100 sequence.txt

Found duplex of 63 bases
nstrands, nnucl =  2 126
Adding duplex of 63 bases
done line 1 / 1, now at 126/126
ALL DONE. just generated 'generated.dat' and 'generated.top'
\end{lstlisting}
\end{itemize}

\end{frame}

\begin{frame}[fragile]
\frametitle{(1) Creating an Initial Configuration}
\small

\begin{itemize}
\item Inspect the oxDNA standalone format: topology file
\linespread{0.4}
\begin{lstlisting}
vi generated.top

126 2             # total no. of nucleotides and strands
1 A -1 1          # strand ID, nucleotide type, 3' partner, 5' partner
1 A 0 2
1 A 1 3
1 A 2 4
...
\end{lstlisting}


\item Inspect the oxDNA standalone format: configuration file
\begin{lstlisting}
vi generated.dat

t = 0                           # timestep
b = 100.0 100.0 100.0           # box dimensions
E = 0. 0. 0.                    # energy 
81.10076700045578 45.544279111797685 26.695973276367443 0.6836324286755338 0.6187606869041726 -0.3870166854349663 0.4842717557006389 0.012139188631672909 0.8748334165599674 0.0 0.0 0.0 0.0 0.0 0.0
81.55953447082155 45.3431814998625 26.89033700942101 0.233605215000666 0.9618090481628894 -0.1426602901878769 0.4842717557006389 0.012139188631672909 0.8748334165599674 0.0 0.0 0.0 0.0 0.0 0.0
...

# position, base vector, base normal, velocity, angular momentum
\end{lstlisting}

\item Rename the files
\begin{lstlisting}
mv generated.top dsDNA_init.top
mv generated.dat dsDNA_init.dat
\end{lstlisting}

\end{itemize}

\end{frame}

\begin{frame}[fragile]
\frametitle{(1) Creating an Initial Configuration}

\begin{itemize}
\item Download the topology file \texttt{dsDNA\_init.top} and configuration file \texttt{dsDNA\_init.dat} into a working directory on your local host.
\begin{lstlisting}
scp username@hostname:/path/to/oxDNA_tutorial/exercises/ \
      1_initial_config/dsDNA_init* /your/working/directory/
\end{lstlisting}
\end{itemize}

\begin{columns}

\begin{column}{0.55\textwidth}

\begin{itemize}
\setlength\itemsep{15pt}
\vspace*{-4.5cm}
\item Open oxView in your browser,\\
\href{https://sulcgroup.github.io/oxdna-viewer}{https://sulcgroup.github.io/oxdna-viewer}\\[3pt]
click on the 'Open' tab,\\[3pt]
navigate to your local working directory where you saved the two files and load them.

\item Inspect and edit the visualisation, e.g. by rotating and translating the dsDNA molecule.
\end{itemize}
\end{column}

\begin{column}{0.4\textwidth}
\includegraphics[width=\textwidth]{oxView_dsDNA_init.png}
\end{column}
\end{columns}

\end{frame}

\begin{frame}[fragile]
\frametitle{(2) Equilibration Using The Standalone Code}
\small
In this exercise we want to equilibrate the initial configuration using the Virtual-Move Monte Carlo (VMMC) algorithm provided in the standalone code.\\[7pt]

\begin{itemize}
\item Copy the topology and configuration file as well as the \texttt{oxDNA} executable into the directory for the second exercise and navigate to this directory.
\begin{lstlisting}
cp /exercises/1_initial_config/dsDNA_init.* \ 
    /exercises/2_MC_equilibration
cp /oxDNA/build/bin/oxDNA /exercises/2_MC_equilibration
\end{lstlisting}
\item Perform a VMMC run with the oxDNA standalone code on a single CPU using the \texttt{input} input file.
\linespread{0.4}
\begin{lstlisting}
./oxDNA input 

     0  -1.679223   0.000  0.000  0.000   
   100  -1.608630   0.751  0.197  0.000   
   200  -1.608018   0.746  0.199  0.000   
   300  -1.587952   0.743  0.200  0.000   

   ...      ...        ...     ...     ...  
\end{lstlisting}
The columns correspond to MC step, potential energy per nucleotide, acceptance ratio for translational moves, acceptance ratio for rotational moves, acceptance ratio for volume moves.

\item The sequence of configurations of the VMMC run is contained in \texttt{dsDNA\_equ.dat}. The last configuration is contained in \texttt{last\_config.dat}.

\end{itemize}

\end{frame}

\begin{frame}[fragile]
\frametitle{(2) Equilibration Using The Standalone Code}

\begin{itemize}
\item Download the configuration file \texttt{dsDNA\_equ.dat} into a working directory on your local host.
\begin{lstlisting}
scp username@hostname:/path/to/oxDNA_tutorial/exercises/ \
      2_MC_equilibration/dsDNA_equ.dat /your/working/directory
\end{lstlisting}
\end{itemize}

\begin{columns}

\begin{column}{0.48\textwidth}

\begin{itemize}
\setlength\itemsep{15pt}
\item In your oxView tab in your browser,\\
click on the 'Open' tab,\\
navigate to your working directory,
\\load the new configuration file and the previous topology file\\
(the topology has not changed).

\item Inspect the output of the VMMC equilibrated dsDNA molecule.
\end{itemize}
\end{column}

\begin{column}{0.48\textwidth}

\includegraphics[width=\textwidth]{oxView_dsDNA_equ.png}

\end{column}
\end{columns}

\end{frame}

\begin{frame}[fragile]
\frametitle{(3) Temperature Quench Using LAMMPS}
\small

The dsDNA molecule has been equilibrated at $T=20^\circ C$. We want to perform a quench to $T=100^\circ C$ with LAMMPS and observe how the dsDNA molecule denatures into two ssDNA molecules.\\[10pt]
In order to do this, we need to convert the last equilibrated configuration \texttt{last\_config.dat} from oxDNA standalone to LAMMPS format.\\[10pt]

\begin{itemize}
\setlength\itemsep{5pt}
\item In your oxView visualisation from exercise (II), check if the dsDNA molecule is centred around the origin. Otherwise navigate to the 'View' tab, select 'Centering $\rightarrow$ Origin' and click 'Apply'.
\item Navigate to the 'File' tab and select 'Downloading the current scene as oxDNA files' from the 'Save' tab. 
\item Export the current topology and configuration file as \texttt{output.top} and \texttt{output.dat} to your Download directory.
\item Navigate to the taxoxDNA webserver at \href{http://tacoxdna.sissa.it}{http://tacoxdna.sissa.it} and select the 'oxDNA $\rightarrow$ LAMMPS' tool.
\item Select \texttt{output.top} and \texttt{output.dat} as oxDNA topology and configuration file and click 'Generate'.
\item Click 'Download output'. A new tab opens in your browser. Save the page as \texttt{txt} file.  
\end{itemize}

\end{frame}

\begin{frame}[fragile]
\frametitle{(3) Temperature Quench Using LAMMPS}

\small

\begin{itemize}
\item Rename the LAMMPS data file to \texttt{data.dsDNA\_equ.lmp} and copy it into the directory for the third exercise on your remote host.

\begin{lstlisting}
scp ./data.dsDNA_equ.lmp username@hostname:/exercises/3_MD_quench
\end{lstlisting}

\item On the remote host open \texttt{data.dsDNA\_equ.lmp} in a text editor.\\
Set the number of bond types to \textbf{\texttt{2 bond types}} and the box boundaries to\\
\textbf{\texttt{-50.0 50.0 xlo xhi}} \textbf{\texttt{-50.0 50.0 ylo yhi}}
\textbf{\texttt{-50.0 50.0 zlo zhi}}.\\
The top of your LAMMPS data file should look like this:
\linespread{0.4}
\begin{lstlisting}
# LAMMPS data file
126 atoms
126 ellipsoids
124 bonds

4 atom types
2 bond types

# System size
-50.000000 50.000000 xlo xhi
-50.000000 50.000000 ylo yhi
-50.000000 50.000000 zlo zhi

    ...          ...         ... 
\end{lstlisting}
The first edit is necessary so we can monitor the potential energy in the FENE backbone interaction. The second edit resets the boundaries with the dsDNA molecule centred in the simulation box.

\end{itemize}

\end{frame}

\begin{frame}[fragile]
\frametitle{(3) Temperature Quench Using LAMMPS}

\begin{itemize}
\item Perform an MD run with LAMMPS on 8 CPUs using the \texttt{in.dsDNA.lmp} input file.
\begin{lstlisting}
mpirun -np 8 ./lmp_mpi -in in.dsDNA.lmp 
LAMMPS (2 Aug 2023 - Update 3)
WARNING: Atom style hybrid defines both, per-type and per-atom masses; both must be set, but only per-atom masses will be used (../atom_vec_hybrid.cpp:132)
Reading data file ...
  orthogonal box = (-50 -50 -50) to (50 50 50)
...
\end{lstlisting}
\item \textbf{Note}: \textbf{On this occasion} do not worry about the warning
\begin{lstlisting}
WARNING: Proc sub-domain size < neighbor skin, could lead to lost atoms
\end{lstlisting}
To optimise runtimes we run here on 8 CPU cores, which leads on average to only around 16 nucleotides per process. At this system size (126 nucleotides), we would normally run only on 2 or 4 MPI-tasks.
\end{itemize}

\end{frame}

\begin{frame}[fragile]
\frametitle{(3) Temperature Quench Using LAMMPS}

Visualising the trajectory with oxView requires transformation from the LAMMPS into the oxDNA format with tacoxDNA.\\[5pt]
As the webserver supports currently only the transformation of data files, tacoxDNA has to be used on the command line.

\begin{itemize}

\item On the remote host clone the tacoxDNA repository from the GitHub website.
\begin{lstlisting}
git clone https://github.com/lorenzo-rovigatti/tacoxDNA
\end{lstlisting}
\item Navigate to your working directory with the LAMMPS trajectory.
\begin{lstlisting}
cd /exercises/3_MD_quench
\end{lstlisting}
\item Transform the LAMMPS trajectory into oxDNA format using the command line. 
\begin{lstlisting}
python /path/to/taxoxDNA/src/LAMMPS_oxDNA.py data.dsDNA.1.lmp \
                                                              out.lammpstrj

Wrote data to 'data.dsDNA.1.lmp.oxdna' / 'data.dsDNA.1.lmp.top'
DONE
\end{lstlisting}
\item Download the topology and configuration file to your local host and visualise the trajectory with oxView.

\end{itemize}


\end{frame}

\begin{frame}[fragile]
\frametitle{(4) oxDNA Analysis Tools}

Calculate the mean structure of a trajectory file
\begin{lstlisting}
oat mean -h

usage: mean.py [-h] [-p num_cpus] [-o output_file] [-d deviation_file] [-i index_file] [-a alignment_configuration]
               trajectory

Computes the mean structure of a trajectory file

positional arguments:
  trajectory            The trajectory file you wish to analyze

optional arguments:
  -h, --help            show this help message and exit
  -p num_cpus           (optional) How many cores to use
  -o output_file, --output output_file
                        The filename to save the mean structure to
  -d deviation_file, --deviations deviation_file
                        Immediatley run oat deviations from the output
  -i index_file         Compute mean structure of a subset of particles from a space-separated list in the provided file
  -a alignment_configuration, --align alignment_configuration
\end{lstlisting}

\end{frame}
\begin{frame}[fragile]
\frametitle{(4) oxDNA Analysis Tools}

\begin{itemize}
\item Calculate the mean structure of VMMC equilibration in \texttt{/exercises/2\_MC\_equilibration}\\
\begin{lstlisting}
oat mean -p 4 -o dsDNA_equ_mean.dat dsDNA_equ.dat 
\end{lstlisting}
\item Visualise with oxView
\end{itemize}

\begin{center}
\includegraphics[width=0.8\textwidth]{oxView_dsDNA_mean_config.png}\\
Requires dsDNA configuration as input for meaningful orientational averaging
\end{center}

\end{frame}

\begin{frame}[fragile]
\frametitle{(4) oxDNA Analysis Tools}

Calculate the ensemble of distances between nucleotides

\begin{lstlisting} 
oat distance -h

usage: distance.py [-h] [-i input [input ...]] [-o output_file] [-f <histogram/trajectory/both>] [-p num_cpus]

Finds the ensemble of distances between any two particles in the system

optional arguments:
  -h, --help            show this help message and exit
  -i input [input ...], --input input [input ...]
                        A trajectory, and a list of particle pairs to compare. Can call -i multiple times to plot
                        multiple datasets.
  -o output_file, --output output_file
                        The name to save the graph file to
  -f <histogram/trajectory/both>, --format <histogram/trajectory/both>
                        Output format for the graphs. Defaults to histogram. Options are "histogram", "trajectory", and
                        "both"
  -p num_cpus, --parallel num_cpus
                        (optional) How many cores to use
\end{lstlisting}


\end{frame}

\begin{frame}[fragile]
\frametitle{(4) oxDNA Analysis Tools}

\begin{itemize}
\item Calculate the ensemble of distances between the first and second and the second and third nucleotide
\begin{lstlisting}
oat distance -p 4 -f both -i data.dsDNA.1.lmp.oxdna 0 1 1 2 -o distance_dsDNA.1.png
\end{lstlisting}
\end{itemize}

\begin{center}
\includegraphics[width=0.49\textwidth]{distance_dsDNA_traj.1.png}
\includegraphics[width=0.49\textwidth]{distance_dsDNA_hist.1.png}\\
\textit{Trajectory of distances}
\hspace*{3cm}
\textit{Histogram of distances}
\end{center}
\end{frame}

\end{document}
